\documentclass[11pt, oneside]{article}   	% use "amsart" instead of "article" for AMSLaTeX format
\usepackage{geometry}                		% See geometry.pdf to learn the layout options. There are lots.
\geometry{letterpaper}                   		% ... or a4paper or a5paper or ... 
%\geometry{landscape}                		% Activate for for rotated page geometry
%\usepackage[parfill]{parskip}    		% Activate to begin paragraphs with an empty line rather than an indent
\usepackage{graphicx}				% Use pdf, png, jpg, or eps§ with pdflatex; use eps in DVI mode
								% TeX will automatically convert eps --> pdf in pdflatex		
\usepackage{amssymb}
\usepackage{amsmath}
\usepackage{gensymb}				% degree symbol
\usepackage{amsthm}
\usepackage{booktabs}				% allows toprule and bottomrule in tables
\usepackage{tabularx}
\usepackage{multirow}				% data spans multiple rows and columns in tables
\usepackage{caption}
\usepackage{setspace}

\usepackage{color}
\newcommand{\edit}{\textcolor{red}}      % highlight text red for editing purposes.

\usepackage[makeroom]{cancel}						% Notation for canceling terms:
\providecommand{\e}[1]{\ensuremath{\times 10^{#1}}} 		% Scientific notation, e.g.: 6.653 \e{-24}


\usepackage{lscape}

\usepackage{fancyheadings}			% customizable headers and footers
\usepackage{lastpage}
\pagestyle{fancy}
\lhead{}
\chead{}
\rhead{}
\lfoot{}
\cfoot{}
\rfoot{\thepage \, of \pageref{LastPage}}

\makeatletter         
\def\@maketitle{   					% custom maketitle 
\begin{center} \Large \bfseries \@title \end{center}
\begin{flushleft} \@author \\ \@date \end{flushleft} \par 
\smallskip \hrule}

\setcounter{secnumdepth}{4}                     % Only number sections, nothing deeper

\title{Personal Library of LaTeX Code \\ } 
\author{Author: John Hayes}
\date{Updated: Tue 8/18/15}							% Activate to display a given date or no date

\begin{document}
\maketitle
\doublespacing
%

\section{Section Header}
\subsection{Subsection Header}
\subsubsection{Subsubsection Header}
\paragraph{Paragraph} 

Label subequations as (1a, 1b ...)
%
\begin{subequations} \label{eqn:Pvalues}
\begin{align} 
P_c &= 5 \, \text{mSv/yr} = 0.1 \, \text{mSv/wk}  \\
P_u &= 1 \, \text{mSv/yr} = 0.02 \, \text{mSv/wk}.
\end{align} 
\end{subequations}
%
Scientific notation in math type $4.997\e{9}$ and in text 3.545\e{7}  

\edit{Highlight text red for editing purposes.} \\

%%%%%%%%%%%%%%%%%
\begin{table}[h!]
\centering
\caption{A sample table with caption above.}
\label{tab:shieldingParams}
\begin{tabular}{|l|c|c|l|}
\hline
Parameter  & Value & Unit & Description \\
\hline 
$d_{pri}$ & 6.71 $\pm$ 0.01 & m & distance from source to 1 m past the primary barrier \\ \hline
$d_{sec}$ & 4.88 $\pm$ 0.01 & m & distance from source to 1 m past the secondary barrier \\ \hline
$d_{sca}$ & 1.0 $\pm$ 0.01 & m & distance from source to phantom \\ \hline
$\dot{X}_U$ & 2458 $\pm$ 4 & R/hr & unattenuated exposure rate 1 m from the source \\ \hline
$\dot{X}_S$ & 1520  $\pm$ 100 & mR/hr & scattered exposure rate 1 m from patient surface \\ \hline
$\dot{X}_L$ & 1800 $\pm$ 300 & $\mu$R/hr & exposure rate due to leakage 1 m from the source \\ \hline
$S_p$ & \textit{to be determined} & m & thickness of primary barrier \\ \hline
$S_s$ & \textit{to be determined} & m & thickness of secondary barrier \\ 
\hline
\end{tabular}
\end{table}
%%%%%%%%%%%%%%%%%
%

\clearpage

\noindent The following are examples of canceling terms. \\ \\
\verb|\cancel{5y}|:
\[ x+\cancel{5y}=0\]
\verb|\bcancel{5y}|:
\[ x+\bcancel{5y}=0\]
\verb|\xcancel{5y}|:
\[ x+\xcancel{5y}=0\]
\verb|\cancelto{\infty}{5y}|:
\[ x+\cancelto{\infty}{5y}=0\]
\verb|\cancelto{0}{5y}|:
\[ x+\cancelto{0}{5y}=0\]
\verb|\cancelto{}{5y}|:
\[ x+\cancelto{}{5y}=0\] \\

\noindent The first three commands work in text mode also i.e., \cancel{5y}, \bcancel{5y} and 
\xcancel{5y}.  But \verb|\cancelto{\infty}{5y}| and other `cancelto' forms do not. 


\clearpage
A reference in the text \cite{NCRP151}.  Reference 2 is not cited in the text but appears in the reference section.

\nocite{NCRP49}
\bibliographystyle{plain}
\bibliography{bibliography}



\end{document}